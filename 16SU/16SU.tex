%%%%%%%%%%%%%%%%%%%%%%%%%%%%%%%%%%%%%%%%%%%%%%%%%%%%%%%%%%%%%%%%%%%%%
% LaTeX Template: Curriculum Vitae
%
% Source: http://www.howtotex.com/
% Feel free to distribute this template, but please keep the
% referal to HowToTeX.com.
% Date: July 2011
%
% This latex file was modified to serve as a semester planner. 
% Author: Camille Avestruz
% Date: August 2016
%%%%%%%%%%%%%%%%%%%%%%%%%%%%%%%%%%%%%%%%%%%%%%%%%%%%%%%%%%%%%%%%%%%%%%
% How to use writeLaTeX: 
%
% You edit the source code here on the left, and the preview on the
% right shows you the result within a few seconds.
%
% Bookmark this page and share the URL with your co-authors. They can
% edit at the same time!
%
% You can upload figures, bibliographies, custom classes and
% styles using the files menu.
%
% If you're new to LaTeX, the wikibook is a great place to start:
% http://en.wikibooks.org/wiki/LaTeX
%
%%%%%%%%%%%%%%%%%%%%%%%%%%%%%%%%%%%%%%%%%%%%%%%%%%%%%%%%%%%%%%%%%%%%%%
\documentclass[paper=a4,fontsize=11pt]{scrartcl} % KOMA-article class

\usepackage[english]{babel}
\usepackage{soul}
\usepackage{hyperref}
\usepackage[utf8x]{inputenc}
\usepackage[protrusion=true,expansion=true]{microtype}
\usepackage{amsmath,amsfonts,amsthm,amssymb}     % Math packages
\usepackage{graphicx}                    % Enable pdflatex
\usepackage[svgnames]{xcolor}            % Colors by their 'svgnames'
\usepackage{geometry}
\textheight=700px                    % Saving trees ;-)
\usepackage{url}

\frenchspacing              % Better looking spacings after periods
\pagestyle{empty}           % No pagenumbers/headers/footers

%%% Custom sectioning (sectsty package)
%%% ------------------------------------------------------------
\usepackage{sectsty}

\sectionfont{%            % Change font of \section command
  \usefont{OT1}{phv}{b}{n}%% bch-b-n: CharterBT-Bold font
  \sectionrule{0pt}{0pt}{-5pt}{3pt}}

%%% Macros
%%% ------------------------------------------------------------
\newlength{\spacebox}
\settowidth{\spacebox}{8888888888888888}% Box to align text
\newcommand{\sepspace}{\vspace*{1.5em}}% Vertical space macro
\newcommand{\dailyspace}{\vspace*{1.0em}}% Vertical space macro
\newcommand{\smallspace}{\vspace*{0.7em}}% Vertical space macro

\newcommand{\MyName}[1]{ % Name
  \Huge \usefont{OT1}{phv}{b}{n} \hfill #1
  \par \normalsize \normalfont}

\newcommand{\MySlogan}[1]{ % Slogan (optional)
  \large \usefont{OT1}{phv}{m}{n}\hfill \textit{#1}
  \par \normalsize \normalfont}

\newcommand{\NewPart}[1]{\section*{\uppercase{#1}}}

\newcommand{\MajorProjectDescription}[2]{
  \vspace{10pt}\noindent\hangindent=0.2em\hangafter=0 % Indentation
    \textbf{#1}\par\vspace{5pt}\par}       % Entry name (birth, address, etc.)


\newcommand{\MajorProjectGoal}[1]{
  \hspace{5pt} #1 \par}    % Entry value

\newcommand{\WeeklyEntry}[2]{
  \vspace{5pt}\noindent \textbf{#1} \hfill      % Study
  \colorbox{Black}{%
    \parbox{25em}{%
      \hfill\color{White}#2}} \par\vspace{5pt}}  % Dates

\newcommand{\DailyEntry}[1]{
  \dailyspace
  \noindent {\textbf{\textit{#1}}} \par\smallspace}  % Day of week and Date

\newcommand{\Project}[1]{
  \smallspace
  \noindent \textit{#1:}\par        % Project
}

\newcommand{\Task}[1]{
  \hspace{3pt} {\small\textbf{#1}} \par\par % Task Description
}

\newcommand{\MiniTask}[1]{
  \hspace{20pt} {\scriptsize$\blacksquare$}\hspace{4pt}{\small #1} \par\par % Task Description
}

\newcommand{\IP}[0]{
  \Project{\textcolor{red}{Image Processing}}
}

\newcommand{\IE}[0]{
  \Project{\textcolor{green}{ICM Evolution}}
}

\newcommand{\Diss}[0]{
  \Project{\textcolor{blue}{Dissertation Revisions}}
}

\newcommand{\Errands}[0]{
  \Project{\textcolor{magenta}{Errands}}
}

\newcommand{\TravelTalks}[0]{
  \Project{\textcolor{brown}{Travel and Talks}}
}

\newcommand{\DoneTask}[1]{
  \hspace{3pt}  {\small\st{\textbf{#1}}} \par\par% Task Description
}

\newcommand{\DoneMiniTask}[1]{
  \hspace{20pt} {\scriptsize$\blacksquare$}\hspace{4pt} {\small\st{#1}}  % Task Description
  \par\par
}


\newcommand{\Key}[2]{      % Same as \PersonalEntry
  \noindent\hangindent=2em\hangafter=0 % Indentation
  \parbox{\spacebox}{        % Box to align text
    \textit{#1}}   % Entry name (birth, address, etc.)
  \hspace{1.5em} #2 \par}    % Entry value



%%% Begin Document
%%% ------------------------------------------------------------
\begin{document}
% you can upload a photo and include it here...
%\begin{wrapfigure}{l}{0.5\textwidth}
%\vspace*{-2em}
%\includegraphics[width=0.15\textwidth]{photo}
%\end{wrapfigure}

\MyName{Camille Avestruz}
\MySlogan{End of Summer 2016 Project Plans}

\sepspace

%%% Major Projects
%%% ------------------------------------------------------------
\NewPart{Major Projects and Goals}{}

\MajorProjectDescription{\textcolor{red}{Image Processing for Strong Lensing}}

\DoneTask{Check that reading in fits is comparable to png}
\Task{Get a better score from grid search parameters}
\MiniTask{Set up on Midway (or orion) - look at ``onboarding'' sheet of paper Valeri scribbled notes on}
\Task{Explore classifiers: Run the pipeline with SVM instead of Logistic Regression}
\Task{Add other image processing components}
\MiniTask{Test each image processing component with a subset of the data}
\Task{Compile the purity and completeness figures for the different combinations of image processing}
\Task{Understand why some images were false positives or false negatives under given combinations of the image processing}
\Task{Write each section up}
\Task{Get code into publicly releasable form}

\MajorProjectDescription{\textcolor{green}{ICM Evolution and Scaling Relations}}

\Task{Automate what I pull out of the code}
\MiniTask{Add to the ``pipeline'' to build the code from the training set}
\MiniTask{Calculate Yth and Ynt for 200m, 500c as a derived field to pull}
\MiniTask{Will have to account for radial quantities vs. integrated quantities in building models}
\Task{Get a predicted Y-M as a function of peak height and redshift}
\MiniTask{Double check the peak height calculation}
\Task{Watch David Hogg's colloquium - how did they frame the problem of a data driven model for stars?  Also, were there outliers from the Main Sequence?  Does that apply to the merger/AGN interplay from the Pontzen+16 paper?}
\Task{Decide what I want out of this paper – what do my results imply?}
\Task{Chat with Andrey to figure out least publishable unit.}
\Task{Begin shaping paper with a summary and story}
\Task{Choose key plots for story}
\Task{Put together the story in words}

\MajorProjectDescription{\textcolor{blue}{Dissertation Revisions}}

\Task{Go through Chapter 2 revisions}
\MiniTask{Something about line of sight dl vs. dV in the EMM 
description?}
\MiniTask{Remove the metallicity figure - can't find colorbar that went 
here.  Maybe add the figure of clumping for NR, CSF, and AGN instead?}
\MiniTask{Can't find figures for the mechanical feedback module tests: rowse through the old google blogger archives.  
Otherwise, take out the mechanical module description(?)}
\MiniTask{Conclude with other works that have extended beyond the BS09 
model: Ramses (the Martizzi and Dubois papers), Choi+Ostriker papers, Steinborn+ in the Munich group, which was based on the Gaspari+.... 
Emphasize that these works provide the context for future work as outlined in Chapter 5.}
\Task{Email people}
\MiniTask{Email Sandy, get other to-dos}
\MiniTask{Email Committee with summary of changes}
\MiniTask{Submit all necessary paperwork}
\Task{Get Proof of Deposit/Completion from Registrar}
\MiniTask{Contact Registrar}
\MiniTask{Forward relevant info to MMiller}

\MajorProjectDescription{\textcolor{magenta}{Errands}}
\Task{Email Gourav relevant papers for ICM scaling}
\MiniTask{Analytic formulation, Dalal+10}
\MiniTask{Non-thermal pressure fraction, Nelson+14b, Avestruz+16}

\DoneTask{KICP one page annual report}

\Task{Feedback for Collaborators}
\MiniTask{Download and look over \href{https://urldefense.proofpoint.com/v2/url?u=https-3A__bitbucket.org_ethlau_eos&d=CwMCaQ&c=-dg2m7zWuuDZ0MUcV7Sdqw&r=gJS8dUZZIuuJT4PmC0HIzER-8ltvgYEa7iI1ZqHFWqQ&m=KIyZYUSEXQh_YCPHD7mLMJRY-2zlck11La--5cznuPg&s=jWpko2c_6XCpxR-K04ogwVvb1O9rbwRhDrkRe-GNF7s&e=}{Erwin's EoS Paper}}
\MiniTask{Watch \href{http://online.kitp.ucsb.edu/online/colduniv16/mccourt/}{McCourt Talk}}
\MiniTask{Look at Phil's paper, especially sec 2.2.3, and App A}

\pagebreak
%%% Weekly Plans
%%% ------------------------------------------------------------
\NewPart{Weekly Plans}{}
%%% ------------------------------------------------------------
\WeeklyEntry{8 Aug 2016 - 12 Aug 2016}{Week 1}
%%% ------------------------------------------------------------
%%% ------------------------------------------------------------
\DailyEntry{Monday, 8 Aug 2016}
%%% ------------------------------------------------------------
\IP
\Task{Check that reading in fits is comparable to png.  We want to use fits because these are floats, instead of unsigned integer (in png)}
\DoneMiniTask{Use getdata from https://pythonhosted.org/pyfits/ to read in fits as 2d array.  \textbf{This is a PrimaryHDU type, and not easy to compare with float. Also, need to index the 0th entry to get the same np.shape of 300,300}}
\DoneMiniTask{Compare the shapes of 2d array from getdata and imread on png}
\DoneMiniTask{Change the piece of code from imread to getdata. \textbf{Actually, changing to the astropy.io.fits.  Should be similar to getdata, but returns a 300,300 numpy array of floats.}}
\DoneMiniTask{Debug problem: ``map.error: [Errno 24] Too many open files''. \textbf{Googled solution is to use .copy because the list appending keeps files open}}
\DoneMiniTask{Issue with reading in fits: ``raise ValueError(``Images of type float must be between -1 and 1.'') ``: https://github.com/scikit-image/scikit-image/issues/1627 $\rightarrow$ \textbf{I just normalized by the maximum in each image}}

\IE
\Task{Figure out a mini list for this}
\DoneMiniTask{Look over git log}
\DoneMiniTask{Map out changes that still need to be made before getting my Y-M relations}
\Task{Get functionality from dev/ to the run\_pipeline.py}
\DoneMiniTask{Get fit\_samples ported over, looking at test\_load\_samples.fit\_samples, port into training\_steps.py}
%%% ------------------------------------------------------------
\DailyEntry{Tuesday, 9 Aug 2016}
%%% ------------------------------------------------------------

\IE
\Task{Get functionality from dev/ to the run\_pipeline.py}
\DoneMiniTask{Test steps through fitting the samples in ipython notebook}
\DoneMiniTask{Port over generate\_model\_profiles, which means taking trained\_models and predicting}


\IP
\Task{Check that reading in fits is comparable to png.  We want to use fits because these are floats, instead of unsigned integer (in png)}
\MiniTask{Possible problem: ``Warning: Possible precision loss when converting from float64 to uint8'' $\rightarrow$ \textbf{Log it first(?)}}
\MiniTask{Run on a subset (e.g. 200) with using getdata for reading fits$\rightarrow$\textbf{See test3.txt, likely overfit, or info loss with uint8.  Will run on all in test4.txt while doing IE.  Log it first(?)}}
\DoneMiniTask{Download sample HST-like fits files from photon}
\DoneMiniTask{In ipython notebook, figure out how to render images of fits files.}
\DoneMiniTask{In ipython notebook read in 2 images}
\MiniTask{Note: Used \href{http://www.astropy.org/astropy-tutorials/FITS-images.html}{astropy} site}

\DoneMiniTask{Render Median then log normal it}
\DoneMiniTask{Log the numpy array then render median - Compare$rightarrow$ Logging, normalizing, then taking the median preserves small scale features.}
\DoneMiniTask{Try with different parameters for median}


%%% ------------------------------------------------------------
\DailyEntry{Wednesday, 10 Aug 2016}
%%% ------------------------------------------------------------
\IP
\Task{Get code to run on appropriate grid search parameters}
\DoneMiniTask{Produce intermediate images after each preprocessing step - e.g. see example in http://scikit-image.org/docs/dev/auto\_examples/applications/plot\_rank\_filters.html}
\DoneMiniTask{Find a good way to visualize the info we get out of HOG - see documentation}
\Task{Get set up on Midway or Orion - 30min}
\DoneMiniTask{Get my password/.ssh key it}
\DoneMiniTask{Pull the code from repo onto server}
\DoneMiniTask{Download fits images from photon into server}


\Errands
\Task{Friday seminar committee}
\DoneMiniTask{Set up the google page}


%%% ------------------------------------------------------------
\DailyEntry{Thursday, 11 Aug 2016}
%%% ------------------------------------------------------------
\IP
\DoneTask{Understand the best normalization for unlensed data.}
\DoneMiniTask{Need to subtract the minimum (to get zero as min), add epsilon or 1 so all are positive definite, then log it, then normalize so it is between -1 and 1}  
\DoneTask{Get set up on Midway or Orion - 30min}
\DoneMiniTask{Download Anaconda on Orion home directory, prepend to PATH}
\DoneMiniTask{Download git to home directory}
\DoneMiniTask{Test bits of code in midway to make sure everything necessary is downloaded}
\Task{Get code to run on appropriate grid search parameters}
\DoneMiniTask{Try with different parameters for HOG, given best params for median - Note: HOG outputs a 1-d histogram}




%%% ------------------------------------------------------------
\DailyEntry{Friday, 12 Aug 2016}
%%% ------------------------------------------------------------

\IE
\Task{Get functionality from dev/ to the run\_pipeline.py}
\DoneMiniTask{Create a module/function to iterate over scaled radii}
\DoneMiniTask{Need to iterate over radii and make predictions.  Then I need to create a profile for each aexp, nu pair (or X product)}


\Diss
\Task{Something about line of sight dl vs. dV in the EMM description?}
\MiniTask{Look up the links I saved}
\MiniTask{Decide how I want to ``correct'' this}

\Errands
\Task{ICM Minnesota Poster}
\DoneMiniTask{Email title/abstract}

\Task{Erwin's EoS paper (30 minutes)}
\DoneMiniTask{Download and look over Erwin's EoS Paper}
\DoneMiniTask{Pull from the repo}
\MiniTask{Send email to meet at 2}
\Task{Phil's paper}
\DoneMiniTask{Download from email}
\DoneMiniTask{2.3.3  (50 minutes)}
\MiniTask{Appendix A}
\MiniTask{Other parts(?)}
\MiniTask{Email Phil and Andrey - make sure to start with positive, then follow with the constructive}

\DoneTask{KICP one page annual report (1 hour en route home)}


%%% ------------------------------------------------------------
\WeeklyEntry{15 Aug 2016 - 19 Aug 2016}{Week 2}
%%% ------------------------------------------------------------
%%% ------------------------------------------------------------
\DailyEntry{Monday, 15 Aug 2016}
%%% ------------------------------------------------------------
\IE
\Task{Get functionality from dev/ to the run\_pipeline.py}
\MiniTask{Port tested profile to the pipeline}
\MiniTask{Add functionality for an integrated quantity, e.g. mass (the model *should* be close to the computed relation, good test - decide how I want to treat integrated quantities.}
\MiniTask{Save the trained models (aexp, nu) pairs - pickle files(?)}

\Task{Get Predicted Y-M Relation}
\MiniTask{}

\IP
\Task{Get code to run on appropriate grid search parameters}
\MiniTask{Look into \href{http://scikit-image.org/docs/dev/api/skimage.feature.html#skimage.feature.hog}{HOG}}
\MiniTask{Better understand what each kwarg setting means: (1) Orientations, (2) pixels\_per\_cell, (3) cells\_per\_block}
\MiniTask{Put the image production into a function that takes in a set of parameters of each}
\MiniTask{Need to check problem with UserWarning: \textit{Possible precision loss when converting from float64 to uint8}}
\MiniTask{Note: Referencing \href{http://scikit-image.org/docs/dev/user_guide/data_types.html}{Data types of skimage}}
\MiniTask{Need to check problem when running on whole dataset with \href{http://stackoverflow.com/questions/30834132/multiprocessing-ioerror-bad-message-length}{Pool}}

\Task{Add in classifiers}
\MiniTask{Look into \href{http://stackoverflow.com/questions/14173128/resize-hog-feature-to-put-in-classifier-scikit-learn}{classifiers}}
\MiniTask{}


\Diss
\Task{Remove the metallicity figure - can’t find colorbar that went here. Maybe add the figure of clumping for NR, CSF, and AGN instead?}

\Diss
\Task{Conclude with other works that have extended beyond the BS09 model}
\MiniTask{Add citations for: Ramses (the Martizzi and Dubois papers), Choi+Ostriker papers, Steinborn+ in the Munich group, which was based on the Gaspari+.... }
\Minitask{Emphasize that these works provide the context for future work as outlined in Chapter 5.}

\Errands
\Task{ICM Minnesota Poster}
\MiniTask{Outline the poster - see A's sample (30 minutes)}
\MiniTask{Fill in each section ..... .... (30 minutes/day for week 2)}




%%% ------------------------------------------------------------
\sepspace
%%% Broad Schedule
%%% ------------------------------------------------------------
\NewPart{Broad Schedule}{}

\Key{Week 1: 8/8-8/12}{\textcolor{green}{ICM MiniList and Y-M}}
\Key{}{\textcolor{red}{Check that reading in fits is comparable to png}}
\Key{}{\textcolor{blue}{Go through Chapter 2 revisions}}
\sepspace
\Key{Week 2: 8/15-8/19}{\textcolor{green}{Decide what I want out of this paper – what do my results imply?}}
\Key{}{\textcolor{red}{Get a better score from grid search parameters}}
\Key{}{\textcolor{blue}{Email people}}
\Key{}{\textcolor{brown}{Make Poster}}
\sepspace
\Key{Week 3: 8/22-8/26}{\textcolor{brown}{Physics of the ICM at U Minn (Travel 8/21-24)}}
\Key{}{\textcolor{brown}{Put together slides for Fellows at the Frontiers}}
\Key{}{\textcolor{blue}{Get Proof of Deposit/Completion from Registrar}}
\sepspace
\Key{Week 4: 8/29-9/2}{\textcolor{red}{Get SVM implemented}}
\Key{}{\textcolor{brown}{Fellows at the Frontiers (Travel 8/31-9/2)}}
\sepspace
\Key{Week 5: 9/6-9/9}{\textcolor{red}{Test SVM parameters}}}
\sepspace
\Key{Week 6: 9/12-9/16}{}
\sepspace
\Key{Week 7: 9/19-9/23}{}

\end{document}


